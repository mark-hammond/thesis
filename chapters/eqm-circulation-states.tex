\begin{SingleSpace}
\chapter{Equilibrium Circulation States on Tidally Locked Planets}\label{ch:eqm-circulation-states}
\vspace{0.5cm}
\chapterprecishere{``Very strange,'' he said. ``A permanent anticyclone, and inside a huge, calm land that never sees a storm and never has a drop of rain.''\\
\\
``Good place for a holiday then!''\par\raggedleft--- \textup{Terry Pratchett}, The Last Continent}
\end{SingleSpace}
\vspace{0.5cm}

%0 -- LEAD-IN PARAGRAPHS

%START ELEMENT
Tidally locked planetary atmospheres

%FRAMING TEXT


%SIGNPOSTS



%SUMMARISE CONCLUSIONS


%SECTION 1 --
\section{Equatorial Acceleration}

Chapter X showed that the global circulation and temperature distribution of an atmosphere on a tidally locked planet depends greatly on the zonal jets present on the planet.


%SUBSECTION --
\subsection*{Acceleration in a Matsuno-Gill Model}

The zonally averaged zonal momentum equation is:

\begin{equation}
  \frac { \partial \overline { u } } { \partial t } = \underbrace { \overline { v } ^ { * } \left[ f - \frac { \partial \overline { u } } { \partial y } \right] } _ { I } \underbrace { - \frac { 1 } { \overline { h } } \frac { \partial } { \partial y } \left[ \overline { ( h v ) ^ { \prime } u ^ { \prime } } \right] } _ { I I } + \underbrace { \left[ \frac { 1 } { \overline { h } } \overline { u ^ { \prime } Q ^ { \prime } } + \overline { R _ { u } } ^ { * } \right] } _ { I I I } \underbrace { - \frac { \overline { u } ^ { * } } { \tau _ { \mathrm { drag } } } } _ { I V } - \frac { 1 } { \overline { h } } \frac { \partial \left( \overline { h ^ { \prime } u ^ { \prime } } \right) } { \partial t }
\end{equation}

%SUBSECTION --
\subsection*{Vertical Momentum Transport}

%SUBSECTION --
\subsection*{Horizontal Momentum Transport from Stationary Waves}

%SUBSECTION --
\subsection*{Horizontal Momentum Transport from Transient Waves}



%SECTION CONCLUSIONS

%SECTION 1.5 --
\section{Midlatitude Acceleration}

Chapter X showed that the global circulation and temperature distribution of an atmosphere on a tidally locked planet depends greatly on the zonal jets present on the planet.

%SUBSECTION --
\subsection*{Acceleration from Rossby Wave Breaking}

%SUBSECTION --
\subsection*{Scaling}

%SUBSECTION --
\subsection*{GCM Simulations}

Figure X shows a suite of tests (P+H2019) showing how the number of jets varies with rotation rate and temperature.

Figure X shows the spin-up of a very rapidly rotating case.




%SECTION CONCLUSIONS




%SECTION 2 --
\section{Initial Conditions}


%SUBSECTION --
\subsection*{Starting from Rest}

%SUBSECTION --
\subsection*{{Initially Retrograde Flow}}

%SUBSECTION --
\subsection*{Initially Strong Prograde Flow}



%SECTION CONCLUSIONS


%
% %SECTION 3 --
% \section{Adjustment in Linear Model}
%
% %SUBSECTION --
% \subsection*{Time-Stepped Linear Model}
%
% %SUBSECTION --
% \subsection*{Mid-Latitude Acceleleration Correction}
%
% %SUBSECTION --
% \subsection*{Equilibrium Jet Speed and Observables}
%
% XX states that the jet speed in their investigation of the observables on a tidally locked planet is not predicted, and is a diagnostic from a GCM. I derive a predicted equilbrium jet speed and velocity profile from the linear theory, and compare it to GCM simulations. I compare it to the simpler on-equator prediction of SP2011.
%
%
% %SECTION CONCLUSIONS



%SECTION 4 --
\section{Instabilities as deviations from equilibrium}

%SUBSECTION --
\subsection*{Linear Model Instability Analysis}

The linear model predicts instabilities.

%SUBSECTION --
\subsection*{Instabilities in GCM}

Instabilities appear in the GCM

%SECTION CONCLUSIONS



%SECTION 5 --
\section{Jet Scaling Relations}

%SUBSECTION --
\subsection*{Equatorial versus Midlatitude Jets}



%SECTION CONCLUSIONS




%CONCLUSIONS

%RESTATE SECTION CONCLUSIONS

%OPEN OUT CONCLUSIONS


% \bibliographystyle{unsrtnat}
% \bibliography{../references.bib}
