\begin{SingleSpace}
\chapter{The Meridional Circulation and Global Superrotation on Tidally Locked Planets}\label{ch:eqm-circulation-states}
\vspace{0.5cm}
\chapterprecishere{``I scarcely need remark that I look upon Laplace's process as a mere sport with symbols, and upon Laplace's conclusion as a grievous error.''\par\raggedleft--- \textupGeorge Biddell Airy}
\end{SingleSpace}
\vspace{0.5cm}

%0 -- LEAD-IN PARAGRAPHS

%START ELEMENT
We can divide the formation of the global circulation into two parts. First, the meridional circulation, which provides the surface easterlies which give the westerly torque on the atmosphere that is necessary to global superrotation. Second, the horizontal momentum transport processes in the jet player -- the stationary waves, transient eddies, and poleward meridional circulation -- which determine the flow profile there, and the presence or absence of equatorial superrotation.

Superrotation index.


%FRAMING TEXT


%SIGNPOSTS



%SUMMARISE CONCLUSIONS


%SECTION 1 --
\section{The Meridional Circulation}

%SUBSECTION --
\subsection*{Axisymmetric}

%SUBSECTION --
\subsection*{Non-axisymmetric}

Vertical transport.

%SUBSECTION --
\subsection*{Tidally locked planets}

Asymmetric, zonal jet, and waves.

%SECTION CONCLUSIONS


%SECTION 2 --
\section{Jet Layer Processes}

%SUBSECTION --
\subsection*{Stationary waves}

%SUBSECTION --
\subsection*{Baroclinic instability and Rossby waves}

%SUBSECTION --
\subsection*{Meridional Circulation}

%SUBSECTION --
\subsection*{Equatorial Jet Direction}

%SECTION CONCLUSIONS


%SECTION 3 --
\section{Discussion}

%SUBSECTION --
\subsection*{Terrestrial and Gaseous Planets}

%SUBSECTION --
\subsection*{Spin-up from different initial conditions}

%SECTION CONCLUSIONS


%CONCLUSIONS
\section{Conclusions}

%RESTATE SECTION CONCLUSIONS

%OPEN OUT CONCLUSIONS


% \bibliographystyle{unsrtnat}
% \bibliography{../references.bib}
