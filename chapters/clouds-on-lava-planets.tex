\begin{SingleSpace}
\chapter{Clouds on Lava Planets}
\vspace{0.5cm}
\chapterprecishere{``One face is forever sunlit, and one forever dark, and only the planet's slow liberation gives the twilight zone a semblance of seasons.''\par\raggedleft--- \textup{Stanley G. Weinbaum}, The Lotus Eaters}
\end{SingleSpace}
\vspace{0.5cm}





% 0 -- LEAD-IN PARAGRAPHS

%START ELEMENT
Cloud-covered exoplanets are a great problem for exoplanet observers, turning illuminating spectra into flat lines. Uniform cloud cover can be an issue, but heterogenous cloud cover may be useful.



%FRAMING TEXT
Hot Jupiters are suggested to have cloud cover. Lava planets could have

%SIGNPOSTS


%SUMMARISE CONCLUSIONS
In this chapter, I address the outstanding question from Chapter X -- could the difference between our model results of 55 Cancri e and the observed low night-side temperature be due to high night-side clouds? I also discuss the effect of clouds on global circulation and on observables such as hot-spot shift.


%SECTION 1 --
\section{Clouds on Lava Planets}


%SUBSECTION --

%SUBSECTION --

%SECTION CONCLUSIONS


%SECTION 2 --
\section{Simulations of Clouds}


%SUBSECTION --

%SUBSECTION --

%SECTION CONCLUSIONS


%SECTION 3 --
\section{Effect on Observations}


%SUBSECTION --

%SUBSECTION --

%SECTION CONCLUSIONS

%CONCLUSIONS

%RESTATE SECTION CONCLUSIONS

%OPEN OUT CONCLUSIONS
