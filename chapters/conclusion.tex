\chapter{Conclusions}\label{ch:conclusions}
%\addcontentsline{toc}{chapter}{\nameref{ch:conclusions}}

% \vspace{0.5cm}

This thesis aimed to understand the atmospheric circulation and global temperature distribution of tidally locked planets. The first two chapters addressed the theory of the formation of the atmospheric circulation, and the last two chapters applied this theory to a case study of the observed thermal phase curve of the planet 55 Cancri e.

%The first two chapters investigated the formation of zonal flow on tidally locked planets, and its interactions with the planetary waves produced by day-night forcing that produced the hot-spot shift. The last two chapters applied this theory to simulate the atmosphere of 55 Cancri e in order to interpret the observed phase curve, using an idealised model to investigate the bulk dynamics, and then a more realistic model to investigate the effect of atmospheric opacity and thickness on the observations.

After introducing the topic of the thesis and reviewing relevant work in Chapters \ref{sec:intro} and \ref{ch:lava-planets}, I investigated the formation of the zonal flow that dominates the atmospheres of tidally locked planets in Chapter \ref{ch:eqm-zonal-flow}. I showed that the meridional circulation is vital to the formation of this flow, and used the Gierasch-Rossow-Williams mechanism to explain how it combines with equatorward momentum transport from stationary waves to produce superrotation. This correctly predicted the equilibrium momentum balance at all latitudes in linear and non-linear shallow-water models and 3D GCM simulations. I used this mechanism to explain the qualitative scaling of the zonal flow of a suite of simulations, and concluded that the equatorial jet dominates the high-latitude jets for strong stellar forcing, and vice versa. Further work could apply this mechanism to predict scaling relations for observables such as zonal flow speed and hot-spot shift, or consider how the mechanism operates in the thicker atmospheres of hot Jupiters.


Chapter \ref{ch:wave-mean-flow} shows how this zonal flow produces the eastward hot-spot shift in the atmospheres of tidally locked planets. I linearised a shallow-water model on an equatorial beta-plane around an equatorial jet and its associated height perturbation. The resulting response to day-night forcing matched the form of the hot-spot shift seen in 3D simulations. It showed that the hot-spot shift is not a result of advection of hot air from the substellar point. Instead, it is a combination of the stationary Rossby and Kelvin waves excited by the forcing, and the zonally uniform pressure perturbation required to geostrophically balance the equatorial jet. This explained how the observable hot-spot shift and day-night contrast scale with planetary properties. These two chapters showed how the zonal flow forms on a tidally locked planet, and how it then affects the observable temperature distribution.


I applied the theory of the first two chapters to a case study of  the observations of the lava planet 55 Cancri e. Chapter \ref{ch:linking-climate-55cnce} used 1D scaling relations and idealised 3D atmospheric simulations to interpret the thermal phase curve measured by \citet{demory201655cnce}. The scaling relations qualitatively predicted the effect of surface pressure and mean molecular weight on the global circulation of the simulations, and on the observable day-night contrast and hot-spot shift. It was possible to match either the observed day-night contrast with a high mean molecular weight atmosphere, or the observed hot-spot shift with a low mean molecular weight and high surface pressure. The ``best-fit'' simulation with intermediate properties selected using the scaling relations had a hot-spot shift of \ang{25} east compared to the observed value of \ang{41}. I concluded that the scaling relations were a powerful tool to predict simulation results and interpret the observations, but that the radiative effects of gaseous absorbers and clouds needed to be represented to properly model the thermal emission.


Motivated by Chapter \ref{ch:linking-climate-55cnce}, Chapter \ref{ch:clouds-lava-planets} used an improved radiative transfer model to investigate the effect of realistic gaseous absorption on the global climate and the thermal emission of 55 Cancri e. I found that atmospheres with absorption dominated by CO behaved similarly to the idealised grey-gas simulations in the previous chapter, suggesting that this is a useful approximation and that the details of the radiative transfer do not greatly affect the global circulation. However, the thermal emission and simulated observations calculated with the more realistic radiative transfer model were very different to the grey-gas model. They matched the day-night contrasts of the previous chapter, but hot-spot shifts present in the temperature field did not appear in the \SI{4.5}{\micro\metre} phase curves. I explained why a sufficiently thick atmosphere is required for a shift in the thermal phase curve, and used a test with 100 bar surface pressure to confirm the presence of a shift at higher pressure. I concluded that the phase curve of \citet{demory201655cnce} is evidence for an atmosphere with surface pressure significantly larger than 10 bar and a mean molecular weight greater than that of H$_{2}$.

These chapters addressed the theory of the global circulation of tidally locked terrestrial planetary atmospheres step by step. The first two chapters provided a mechanism for the formation of zonal flow, and a description of the effect of the flow on the observable temperature distribution. They showed that the meridional circulation is vitally important to the momentum budget of the equilibrium zonal flow, and that the hot-spot shift is a combination of stationary waves with the pressure structure of the zonal flow itself. The last two chapters showed how this theory applies to a real planet, and how the thermal phase curve of  55 Cancri e constrains its atmospheric composition. I concluded that a sufficiently thick atmosphere is needed to observe the hot-spot shift, and that a sufficiently high mean molecular weight is required to explain the day-night contrast.

The next steps are to use the theory of the first two chapters to make predictions for observations, such as the dependence of the hot-spot shift and jet speed on planetary parameters like rotation rate and temperature. I will investigate the free modes in the shallow-water models in more details, and relate these to the atmospheric variability in simulations \citep{pierrehumbert2018review} and observations \citep{demory2015variability, armstrong2017variability}. Further work on the phase curves of 55 Cancri e and other planets would benefit from more detailed 3D modelling, which could include the effects of chemistry and cloud formation.


This thesis has developed a new description of the formation and observable effect of the global circulation of tidally locked terrestrial planets. The study of exoplanetary atmospheres is still very young, and has allowed the work in this thesis to be touched by a great range of stimulating ideas, techniques, and (occasional) real data. I hope that the huge potential of exoplanetary science can be realised, and that there are many more surprises to come.


% \bibliographystyle{unsrtnat}
% \bibliography{../references.bib}
