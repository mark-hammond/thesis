\begin{SingleSpace}
\chapter{Lava Planets and 55 Cancri e}
\vspace{0.5cm}
\chapterprecishere{``One face is forever sunlit, and one forever dark, and only the planet's slow liberation gives the twilight zone a semblance of seasons.''\par\raggedleft--- \textup{Stanley G. Weinbaum}, The Lotus Eaters}
\end{SingleSpace}
\vspace{0.5cm}




% 0 -- LEAD-IN PARAGRAPHS

%START ELEMENT
Perhaps the most exciting discovery from the field of exoplanet science is that other stars host planets which are very different from those in our solar system. There are similar planets to those in the solar system -- ``Hot Jupiters'', high-temperature Jupiter-sized gas giants in short-period orbits, or ``Mini-Neptunes'', which show the literal-mindedness of planetary scientists. But some exoplanets have no analogues in the solar system, and ``lava planets'' are some of the best examples of these.


%FRAMING TEXT
This short chapter describes the class of ``lava planets'', particularly the planet 55 Cancri e, and discusses the question that this thesis aims to answers about this planet.

%SIGNPOSTS
I will describe lava planets in general, and list the known planets in this class. I will then discuss the 55 Cancri system, and the lava planet 55 Cancri e in that system.

%SUMMARISE CONCLUSIONS
I will try to show that lava planets are a potentially bountiful area for scientific work, being interesting systems that have observational advantages. I will set up the question of the atmosphere and atmospheric circulation of 55 Cancri e, and show how it relates to the broader question of the nature of the climate of tidally locked planets.


%SECTION 1 -- EXOPLANETS
\section{Exoplanets}

Exoplanets are planets orbiting stars other than our Sun. As far as we know, there is nothing fundamental to distingush the planets in our Solar System from those elsewhere, so it is possible that this specific nomenclature may eventually disappear. I will use the word ``exoplanet''  when discussing specific planets or issues related to their distance, and ``planet'' in a more general or idealised context (such as the first sentence of this paragraph).

There is no better way to date a piece of writing on exoplanets than by announcing how many have been discovered, so I will just note that we know of several thousand and anticipate many more to come. The number of exoplanets which are favourable for detailed observations is still quite small, and we can observe atmospheric details for perhaps only a few dozen planets. In fact, while the title of this thesis suggests it looks at ``lava planets'', there is really only one that is currently observable -- 55 Cancri e. Despite this, I hope to draw general conclusions about the circulation of many types of planet, and contribute to an understanding of tidally locked planets and lava planets for future observations.


%SUBSECTION -- DISCOVERY
\subsection*{Discovering Exoplanets}

This is not a thesis on discovering exoplanets, although the methods of discovery are sometimes relevant to the characterisation that is of interest. Most exoplanets discovered to date have been found using either a ``radial velocity`` method or a ``transit'' method.

In the first method, the motion of a star around its common center of mass with a planet orbiting it is detected by measuring the Doppler-shift of emission lines of the star. The magnitude and period of this motion gives the period of the planet's orbit, and a limit on its mass.

In the second method, a planet passing across the line of sight from an observer to the star produces a dip in the light seen by the observer. A periodic dip gives the period of the planet, and the size of the dip gives its radius. So, if a planet can be measured with both methods the observer retrieves its period, mass, radius, density, semi-major axis, and equilibrium temperature.


%SUBSECTION -- CHARACTERISATION
\subsection*{Characterising Exoplanets}

This is also not a thesis on characterising exoplanets, although I have tried to keep observations in mind throughout the simulations and theory.

The atmospheres of exoplanets can be characterised through transmission and emission spectroscopy. In transmission spectroscopy, light from the host star passes through the atmosphere of the exoplanet before it reaches the observer, and the spectrum is measured. An alternative (but equivalent) view is that the planet appears to have a different radius as it transits its star at different wavelengths -- at a wavelength the atmosphere is more opaque to, the planet appears larger -- so the absorption spectrum of the gases in the atmosphere can be retrieved.

In emission spectroscopy, the spectrum of the light emitted thermally by the planet and its atmosphere is measured. Hotter planets emit more light in this way, so are better suited to this method.

%SECTION CONCLUSIONS




%SECTION 2 -- LAVA PLANETS
\section{Lava Planets}

%SUBSECTION --
\subsection{Tidally Locked Planets}

A tidally locked planet, or a ``synchronously rotating'' planet, always presents the same face to the star it orbits, as its rotation period is the same as its orbital period. An asynchronously rotating planet like the Earth has a different rotation period (1 day) to its orbital period (1 year). Tidal forces slow down the rotation of such planets, until they become tidally locked. The time for a planet to become tidally locked is approximately:


See Chapter \ref{ch:wave-mean-flow} for an investigation of the atmospheric dynamics of tidally locked planets.

Tidally locked planets include Hot Jupiters, Earth-like planets like those in the Trappist-1 system, and lava planets like 55 Cancri e, discussed next.



%SUBSECTION --
\subsection{Lava Planets}

``Lava Planets'' are terrestrial (rocky, not gaseous) planets orbiting very close to their parent star.

%SECTION CONCLUSIONS

%SECTION 3 -- 55 CANCRI E
\section{55 Cancri e}

55 Cancri e is a tidally locked lava planet orbiting the binary star 55 Cancri, 41 light years away in the constellation of Cancer.

%SUBSECTION -- 55 Cancri system
\subsection*{The 55 Cancri system}

Figure X shows the 55 Cancri system.

%SUBSECTION -- 55 Cancri e
\subsection*{55 Cancri e}

55 Cancri e is the closest planet to the G-star 55 Cancri A.

%SUBSECTION -- PHASE CURVE
\subsection*{A Thermal Phase Curve of 55 Cancri e}

A phase curve is the light measured from a planet as it orbits its star. They are particularly useful for tidally locked planets. Figure X shows a phase curve for X.

A thermal phase curve refers to the light emitted by the planet itself, rather than the light it reflects from the star it orbits. For a tidally locked planet, the thermal phase curve shows the hemisphere-averaged brightness temperature of the planet as it rotates.

\citet{demory201655cnce} measured a thermal phase curve of the planet 55 Cancri e.



%CONCLUSIONS

55 Cancri e is currently the most easily observable terrestrial tidally locked planet. Its composition, atmosphere, and circulation provide tests of theories of planet formation and atmospheric dynamics. In this thesis, I will use it as a case study for the atmospheric dynamics of tidally locked planets.

%RESTATE SECTION CONCLUSIONS

%OPEN OUT CONCLUSIONS

% \bibliographystyle{unsrtnat}
% \bibliography{../references.bib}
