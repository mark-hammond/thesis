\begin{abstract}
\OnehalfSpacing

\noindent

Exoplanets are planets orbiting stars other than the Sun. Tidally locked exoplanets always present the same face to their host stars, resulting in a global atmospheric circulation unlike anything in the Solar System. This thesis develops theories of the formation of this global circulation, demonstrates the circulation in atmospheric simulations, and shows its effect on observations of the ``lava planet'' 55 Cancri e.

The first two chapters introduce the topic and review relevant literature. The next two chapters investigate the atmospheric dynamics of terrestrial tidally locked planets. Chapter \ref{ch:eqm-zonal-flow} uses the Gierasch-Rossow-Williams mechanism to describe the formation of zonal flow on these planets. Chapter \ref{ch:wave-mean-flow} uses a shallow-water model linearised about an equatorial jet to show how this zonal flow produces the global circulation pattern and hot-spot shift seen in observations and simulations.

The following chapters use the theory developed in the first part of the thesis alongside atmospheric simulations to interpret the thermal phase curve of 55 Cancri e. Chapter \ref{ch:linking-climate-55cnce} models possible atmospheres with different bulk properties on this planet, and simulates observations for comparison with the original phase curve. Chapter \ref{ch:clouds-lava-planets} simulates similar atmospheres using more realistic radiative transfer, and shows that the global circulation is similar but that the observed hot-spot shift requires a sufficiently thick atmosphere.

I conclude that the meridional circulation is vital to the formation of zonal flow on tidally locked planets, and that the observable hot-spot shift is caused by wave-mean flow interactions between this zonal flow and the stationary wave response to day-night forcing. The case study of 55 Cancri e suggests that the observations are evidence for an atmosphere thicker than 10 bar with a mean molecular weight higher than that of H$_{2}$, with night-side cloud formation. Future work should use this theoretical understanding of the global circulation alongside more realistic modelling, to interpret the more detailed observations that will be made in the coming decades.

\DoubleSpacing

%
%
% I conclude that the meridional circulation is a vital part of the formation of zonal flow on these planets, and that the hot-spot shift formed by this flow is caused by the stationary wave response to day-night forcing. The simulations show that the observed thermal emission of 55 Cancri e requires an atmosphere of over 10 bar surface pressure with a heavier molecular weight than H$_{2}$. I suggest that an understanding of the circulation of these planets is crucial to interpreting observations and understanding their climates.
%
%
%
%   %%%%%
%
% Exoplanets orbit stars other than the Sun and have revealed a wide variety of new types of planetary atmosphere. Tidally locked planets always present the same face to their host stars, resulting in a global circulation unlike any in the Solar System. This thesis develops models and theory of the atmospheric circulation and global temperature distribution of tidally locked terrestrial planets, and uses these to interpret observations of the thermal emission of the tidally locked ``lava planet'' 55 Cancri e.
%
% The first two chapters of results address the atmospheric dynamics of tidally locked planets. Chapter \ref{ch:eqm-zonal-flow} applies the Gierasch-Rossow-Williams mechanism to describe the formation of zonal flow on these planets. Chapter \ref{ch:wave-mean-flow} uses a shallow-water model linearised about an equatorial jet to show how this zonal flow produces the global circulation pattern and hot-spot shift seen in observations and simulations.
%
% The last two chapters of results apply this theoretical basis to interpret observations of 55 Cancri e. Chapter \ref{ch:linking-climate-55cnce} simulate possible atmospheres on this planet with different bulk properties and uses the resulting thermal emission distribution, or ``thermal phase curve'', to constrain the atmospheric properties. Chapter \ref{ch:clouds-lava-planets} simulates similar atmospheres using more realistic radiative transfer, and shows that the global circulation is similar but that the observed hot-spot shift requires a sufficiently thick atmosphere.
%
% I conclude that the meridional circulation is a vital part of the formation of zonal flow on these planets, and that the hot-spot shift formed by this flow is caused by the stationary wave response to day-night forcing. The simulations show that the observed thermal emission of 55 Cancri e requires an atmosphere of over 10 bar surface pressure with a heavier molecular weight than H$_{2}$. I suggest that an understanding of the circulation of these planets is crucial to interpreting observations and understanding their climates.


\end{abstract}
